%% Macros for AMG notes
%\newcommand{\triplenorm}[1]{\ensuremath{| \! | \! | #1 | \! | \! |}}

%% \mathcal for lower case letters
\DeclareMathAlphabet{\mathpzc}{OT1}{pzc}{m}{it}
% If you want to make the resulting characters smaller, use this instead:
%\DeclareFontFamily{OT1}{pzc}{}
%\DeclareFontShape{OT1}{pzc}{m}{it}{<-> s * [0.900] pzcmi7t}{}
%\DeclareMathAlphabet{\mathpzc}{OT1}{pzc}{m}{it}

\newcommand{\blankpage}{\includepdf[pages=-]{6DL/blankline.pdf}}
\newcommand{\newbreak}{\newpage}
\newcommand{\dnn}{{}_n{\rm N}}
\newcommand{\triplenorm}[1]{%
  \left\vert\kern-0.9pt\left\vert\kern-0.9pt\left\vert #1
  \right\vert\kern-0.9pt\right\vert\kern-0.9pt\right\vert}
\newcommand{\Tscalar}[2]{\ensuremath{(#1 , #2)_{\bar R^{-1}}}} %%% macro for star norm.
\newcommand{\Tnorm}[1]{\ensuremath{\|#1\|_{\bar R^{-1}}}}
\newcommand{\Dscalar}[2]{\ensuremath{( #1 , #2 )_D}}
\newcommand{\Dnorm}[1]{\|#1\|_{D}}
\newcommand{\Tproj}{\ensuremath{Q_c}}
\newcommand{\Dproj}{\ensuremath{Q_D}}
\newcommand{\Anorm}[1]{\|#1\|_A}
\newcommand{\trace}{\ensuremath{\operatorname{trace}}}
\newcommand{\Span}{\ensuremath{\operatorname{span}}}

\newcommand{\eqqsim}{\mathbin{\rotatebox[origin=c]{180}{\ensuremath{\cong}}}}

\newcommand{\rs}{\ensuremath{R_s}}

\newcommand{\Vhf}{V_{\rm hf}}
\newcommand{\Vf}{V_{f}}
\newcommand{\Vc}{V_{c}}
\newcommand{\sparse}{\ensuremath{\operatorname{S}}}
\newcommand{\dof}{\ensuremath{N}}
\newcommand{\co}{\ensuremath{C_{o}}}
\newcommand{\Nepo}{{Nepomnyaschikh}}
\renewcommand{\v}{{\cal V}}
% ---------------------------
% packages
% ---------------------------
% AMS math symbol
\usepackage{amsmath}
\usepackage{amsbsy}
\usepackage{amsfonts}
\usepackage{amssymb}
\usepackage{amscd}
\usepackage{mathrsfs}
%---------------------------by shaobo
\usepackage{palatino}
\usepackage{tikz}
\usetikzlibrary{shapes.geometric, arrows}

%---------------------------by shaobo 
%\usepackage{mathptmx}       % selects Times Roman as basic font
%\usepackage{algorithmic}
\usepackage{bm}
%\usepackage{wsuipa}

\usepackage{algpseudocode}
\usepackage[Algorithm]{algorithm}
%% one line ifs and fors commands only for this section
\algnewcommand{\IIf}[1]{\State\algorithmicif\ #1\ \algorithmicthen}
\algnewcommand{\EElse}{\unskip\ \algorithmicelse\ }
\algnewcommand{\EndIIf}{\unskip\ \algorithmicend\ \algorithmicif}
\algnewcommand{\FFor}[1]{\State\algorithmicfor\ #1\ }
\algnewcommand{\EndFFor}{\unskip\ \algorithmicend\ \algorithmicfor}
%%%%%%%%%%%%%%%%%%%%%%%%%%%%%%%%%%%%%

% text fonts
\usepackage{txfonts}
%\usepackage{helvet}         % selects Helvetica as sans-serif font
%\usepackage{courier}        % selects Courier as typewriter font
\usepackage[latin1]{inputenc}

% graphcis
\usepackage{graphicx}
\usepackage{subfigure} %commented by Ludmil, uncommented by Hongxuan
\usepackage{float}

\usepackage{tikz}   %commented by Ludmil, uncommented by Hongxuan
\usetikzlibrary{shapes,arrows,decorations.pathmorphing,backgrounds,positioning,fit,matrix,calc}  %commented by Ludmil, uncommented by Hongxuan
%\usepackage{subfig}  %added by Ludmil, commented by Hongxuan

\tikzstyle{decision} = [diamond, draw, fill=blue!20,
    text width=4.5em, text badly centered, node distance=3cm, inner sep=0pt]
\tikzstyle{block} = [rectangle, draw, fill=blue!20,
    text width=5em, text centered, rounded corners, minimum height=4em]
\tikzstyle{line} = [draw, -latex']
\tikzstyle{cloud} = [draw, ellipse,fill=red!20, node distance=3cm,
    minimum height=2em]



%% command shell environment 
\usepackage[most]{tcolorbox}
\newtcblisting{commandshell}{colback=black,colupper=white,colframe=yellow!75!black,
listing only,listing options={language=sh},
every listing line={\textcolor{red}{\small\ttfamily\bfseries Terminal \$> }}}

%% Code styles
\usepackage{listings}
\usepackage{color}
 
\definecolor{codegreen}{rgb}{0,0.6,0}
\definecolor{codegray}{rgb}{0.5,0.5,0.5}
\definecolor{codepurple}{rgb}{0.58,0,0.82}
\definecolor{backcolour}{rgb}{0.95,0.95,0.92}
 
\lstdefinestyle{python}{
    backgroundcolor=\color{backcolour},   
    commentstyle=\color{codegreen},
    keywordstyle=\color{magenta},
    numberstyle=\tiny\color{codegray},
    stringstyle=\color{codepurple},
    basicstyle=\footnotesize,
    breakatwhitespace=false,         
    breaklines=true,                 
    captionpos=b,                    
    keepspaces=true,                 
    numbers=left,                    
    numbersep=5pt,                  
    showspaces=false,                
    showstringspaces=false,
    showtabs=false,                  
    tabsize=2
}
 
%\lstset{style=python}


% other tools
\usepackage{hyperref}
\usepackage{type1cm}
%%\usepackage[normalem]{ulem}
\usepackage{soul}
%\usepackage{times}
\usepackage{url}
\usepackage{undertilde}
%\usepackage{ulem}
%\def\utilde{\uwave}
\usepackage{rotating} %% commented because it crashes "figure" (ludmil), uncommented by Hongxuan
\usepackage{makeidx}
\makeindex             % used for the subject index
\usepackage{multicol}
\usepackage{enumerate}
\usepackage{xspace}
 \usepackage{comment}  % added by Ludmil
 % \usepackage[bw,framed]{mcode} %% commented because it crashes "lstlisting" (ludmil)
\usepackage[margins]{trackchanges} %% commented because it crashes "figure" (ludmil), uncommented by Hongxuan
\usepackage{etoolbox}
\usepackage{array}
%%
% ---------------------------
% Formats
% ---------------------------
% head
\usepackage{fancyhdr}
\pagestyle{fancy}
\rhead{} %added by Hongxuan Zhang

% page
\pagenumbering{arabic}
\newcommand{\pinput}[1]
{\newpag
\hrule
\centerline{\huge \bf #1}
\hrule
\newpage}

% ---------------------------
% Theorems etc.
% ---------------------------
\ifcsmacro{theorem}{}{
\newtheorem{theorem}{Theorem}[section]
}
\ifcsmacro{lemma}{}{
\newtheorem{lemma}[theorem]{Lemma}
}
\ifcsmacro{corollary}{}{
\newtheorem{corollary}[theorem]{Corollary}
}
\ifcsmacro{proposition}{}{
\newtheorem{proposition}[theorem]{Proposition}
}
\ifcsmacro{algorithm}{}{
\newtheorem{algorithm}[equation]{Algorithm}
}
\newtheorem{consequence}[equation]{Consequence}
\newtheorem{conclusion}[equation]{Conclusion}
\newcommand{\Remark}{\noindent{\bf Remark.~}}
\newtheorem{assumption}[equation]{Assumption}
\newtheorem{observation}[equation]{Observation}
%\newtheorem{remark}[theorem]{Remark}
%\newtheorem{definition}[theorem]{Definition}
%\newtheorem{exercise}[theorem]{Exercise}
\newtheorem{homework}[theorem]{Homework}
%\newtheorem{example}[theorem]{Example}

\def\proof{\par{\it Proof}. \ignorespaces}
\def\endproof{{\ \vbox{\hrule\hbox{\vrule
        height1.3ex\hskip0.8ex\vrule}\hrule}}\par}

% ---------------------------
% Chapter and sections
% ---------------------------
\let\oldchapter\chapter
\def\chapter{%
  \setcounter{exercise}{0}%
  \oldchapter
}

% ---------------------------
% New commands
% ---------------------------
% References and Citation
%\newcommand{\Label}[1]{\label{#1}{{\mbox{\small\mbox{\fbox{\tt #1}\quad}}}}}
\newcommand{\Label}{\label}
\newcommand{\Rf}[1]{\mbox{$(\ref{#1})$}}
\newcommand{\rf}[1]{$(\ref{#1})$}

\DeclareMathOperator*{\argmin}{arg\,min}
\DeclareMathOperator*{\argmax}{arg\,max}
\DeclareMathOperator*{\range}{Range}
%\DeclareMathOperator*{\span}{span}

% making comment
% \renewcommand{\initialsOne}{Xu}
% \renewcommand{\initialsTwo}{XHu}
% \renewcommand{\initialsThree}{Yang}
% \renewcommand{\initialsFour}{Zhang}
% \renewcommand{\initialsFive}{KHu}


%\newcommand{\sh}{{\cal S}_h}
% new math command

\newcommand{\riz}[1]{\mathpzc{#1}}
%\newcommand{\ritz}[1]{\bar{#1}}
%\newcommand{\ritz}[1]{\cal{#1}}
\newcommand{\ritz}[1]{\mathcal{#1}}
%\newcommand{\ritz}[1]{\underbar{#1}}
%\newcommand{\ritz}[1]{{#1}_r}
%\newcommand{\ritz}[1]{{#1}^r}

\newcommand{\pro}{{\mathcal P}}
\newcommand{\beas}{\begin{eqnarray*}}
\newcommand{\eeas}{\end{eqnarray*}}
\newcommand{\bary}{\begin{array}}
\newcommand{\eary}{\end{array}}
\newcommand{\supp}{{\rm supp}\;}
\newcommand{\update}{\leftarrow}
\newcommand{\cequiv}{\stackrel{\mathrm{c}}{\equiv}}
\newcommand{\hf}{\frac{1}{2}}
\newcommand{\qall}{\quad\forall\;}
\def\ec{\mathrel{\hbox{$\copy\Ea\kern-\wd\Ea\raise-3.5pt\hbox{$\sim$}$}}}
\newcommand{\lc}{\mathrel{\raise2pt\hbox{${\mathop<\limits_{\raise1pt\hbox{\mbox{$\sim$}}}}$}}}
\newcommand{\gc}{\mathrel{\raise2pt\hbox{${\mathop>\limits_{\raise1pt\hbox{\mbox{$\sim$}}}}$}}}
\newcommand{\deq}{\stackrel{\rm def}{=}}
\newcommand{\cths}{\{\cth: h\in\aleph\}}
\newcommand{\td}[1]{\tilde{#1}}
\newcommand{\spd}{SPD }
\newcommand{\step}[1]{\noindent\raisebox{1.5pt}[10pt][0pt]{\tiny\framebox{$#1$}}\xspace}
\newcommand{\topic}[1]{\vspace{2mm}\addtocounter{equation}{1}{\noindent{\bf(\theequation) \ {\sf #1}.}}\ }
\newbox\Ea
\setbox\Ea=\hbox{\raise0.9pt\hbox{$=$}}
\newcommand{\smt}[1]{\overline{#1}}%% Notation for matrix, to be modified.
\newcommand{\rep}[1]{{\widetilde {#1}}}%% Notation for matrix representation
\newcommand{\mt}{\mathcal}%% Notation for matrix, to be modified.
\newcommand{\hcomment}[1]{\mbox{\quad (#1)}}
\newcommand{\tol}{\mbox{\rm tol}}
\newcommand{\diag}{\mbox{diag}\;}
\newcommand{\for}{\sf for\mbox{$\;$}}
\newcommand{\efor}{\mbox{\sf endfor}}
\newcommand{\ot}[1]{\mbox{\bf{$#1$}}}
\newcommand{\hset}{{\aleph}}
\newcommand{\thset}{\{{\cal T}_h: h\in \hset\}}
\newcommand{\mbb}{\mathbb}
\newcommand{\bs}{\boldsymbol}
\newcommand{\mcal}{\mathcal}
\newcommand{\mrm}{\mathrm}
\newcommand{\dist}{\mbox{dist}\;}
\newcommand{\prt}[1]{\frac{\partial}{\partial #1}}
\newcommand{\prtt}[2]{\frac{\partial #1}{\partial #2}}
\newcommand*{\ang}[1]{\left\langle #1 \right\rangle}
\newcommand{\bproof}{\begin{proof}}
\newcommand{\eproof}{\end{proof}}

% new letters
\newcommand{\nne}{\mbox{$n_{\mbox{\tiny E}}$}}
\newcommand{\At}{\mbox{$A_{\mbox{\tiny T}}$}}
\newcommand{\rhst}{\mbox{${f}_{\mbox{\tiny T}}$}}
\newcommand{\rhs}{\mbox{$f$}}
\newcommand{\nt}{\mbox{$NT$}}
\newcommand{\ib}{\mbox{$IB$}}
\newcommand{\nn}{\mbox{$n$}}
\newcommand{\om}{\Omega}  % needs to get rid of it
\newcommand{\Om}{\Omega}
\newcommand{\cM}{{\cal M}}
\newcommand{\m}{\mbox{$\cM$}}
\newcommand{\cT}{\mbox{${\cal T}$}}
\newcommand{\cP}{{\mathcal P}}
\newcommand{\ct}{{\cal T}}  % need to get rid of it
\newcommand{\cth}{{\cal T}_h}
\newcommand{\Th}{\mbox{${\cal T}_h$}}
\newcommand{\ld}{\lambda}
\newcommand{\nh}{{\cal N}_h}
%%%\renewcommand{\v}{{\cal V}}
\newcommand{\vvv}{{\cal V}}
\newcommand{\vh}{{\cal V}_h}
\newcommand{\newu}{u^{new}}
\newcommand{\oldu}{u^{\rm old}}
\newcommand{\oldr}{r^{\rm old}}
\newcommand{\LL}{L}
\newcommand{\Sh}{{\cal S}^h}
\newcommand{\Shz}{{\cal S}^h_0}
\newcommand{\V}{V}
\newcommand{\Q}{Q}
\newcommand{\Ddivh}{\mbox{\bf A}_h^{\rm div}}
\newcommand{\hb}{\hat B}
\newcommand{\tqk}{\tilde Q_k}
\newcommand{\vaip}{\overline{a}}
\newcommand{\Prol}{\Pi}
\newcommand{\Aa}{\bar A}

% operators
\newcommand{\grad}{{\rm grad}}
\newcommand{\curl}{{\rm curl }}
\newcommand{\dv}{{\rm div}}
\newcommand{\divg}{{\rm div}\mbox{$\;$}}
\newcommand{\Hcurl}{{\rm curl}}
\newcommand{\Hg}{H({\rm grad})}

% projections
\newcommand{\Phz}{\Pi_h^0}
\newcommand{\Phg}{\Pi_h}
\newcommand{\Phd}{\Pi_h^{\rm div}}
\newcommand{\PHdiv}{\Pi_H^{\rm div}}
\newcommand{\Phc}{\Pi_h^{\rm curl}}
\newcommand{\PHc}{\Pi_H^{\rm curl}}
\newcommand{\PHcurl}{\Pi_H^{\rm curl}}
\newcommand{\PHd}{\Pi_H^{\rm div}}
\newcommand{\PHz}{\Pi_H^{\rm 0}}

% inner product, norm
\newcommand{\inner}{\mbox{$(\cdot, \cdot)$}}
\newcommand{\innerA}{(\cdot,\cdot)_A}
\newcommand{\nm}[2]{\|{#1}\|_{#2}}
\newcommand{\nmA}[1]{\nm{#1}{A}}
\newcommand{\NA}[1]{\|#1\|_{A}}
\newcommand{\nmm}[1]{\|#1\|}
\newcommand{\trnm}[2]{|\!\!|\!\!|{#1}|\!\!|\!\!|_{#2}}

% space
\newcommand{\ho}{H^1_0(\Om)}
\ifcsmacro{R}{}{
\newcommand{\R}{\mathbb{R}} %% this does not work as \R^{blah blah} better do not use (--ltz)
}
\newcommand{\Z}{\mathbb{Z}}
%\newcommand\Rn[1]{\R^{#1}}
\newcommand{\Hd}{\mbox{$\ot H({\rm div})$}}
\newcommand{\Hdiv}{\mbox{$\ot H({\rm div})$}}
\newcommand{\Hhdiv}{\mbox{$\ot H_h({\rm div})$}}
\newcommand{\HHdiv}{\mbox{$\ot H_H({\rm div})$}}
\newcommand{\Zdiv}{\mbox{$\ot Z({\rm div})$}}
\newcommand{\Zzdiv}{\mbox{$\ot Z_0({\rm div})$}}
\newcommand{\Zhdiv}{\mbox{$\ot Z_h({\rm div})$}}
\newcommand{\ZHdiv}{\mbox{$\ot Z_H({\rm div})$}}
\newcommand{\Hc}{\mbox{$\ot H({\rm curl})$}}
\newcommand{\Hcllurl}{\mbox{$\ot H({\rm curl})$}}
\newcommand{\Hzcurl}{\mbox{$\ot H_0({\rm curl})$}}
\newcommand{\Hhcurl}{\mbox{$\ot H_h({\rm curl})$}}
\newcommand{\HHcurl}{\mbox{$\ot H_H({\rm curl})$}}
\newcommand{\Hhg}{\mbox{$H^1_h$}}
\newcommand{\Hhd}{\mbox{$H_h^{\rm div}$}}
\newcommand{\Hhc}{\mbox{$H_h^{\rm\small curl}$}}
\newcommand{\Hhz}{\mbox{$L^2_h$}}
\newcommand{\Cinf}{C^\infty}
\newcommand{\cc}{C(\bar \Omega)}
\newcommand{\ccc}{C^{0,\lambda}(\bar \Omega)}
\newcommand{\hoz}{H_0^1(\Om)}
\newcommand{\HHg}{\mbox{$H_H^1$}}
\newcommand{\czi}{C_0^\infty(\Omega)}
\newcommand{\domega}{{\cal D}(\Omega)}
\newcommand{\ddomega}{{\cal D}'(\Omega)}
\newcommand{\nmt}[2]{|\!|\!|#1|\!|\!|_{#2}^{\sim}}
\newcommand{\nmaa}[1]{\|#1\|_{0,(\alpha)}}
\newcommand{\nmg}[2]{\nm{#1}{H^{#2}(\Om)}}
\newcommand{\ucc}{\nm{u}{0,\infty}}
\newcommand{\uccc}{\nm{u}{\ccc}}
\newcommand{\unpp}{\nm{u}{W^{d/p,p}(\Omega)}}
\newcommand{\eps}{\epsilon}
\newcommand{\la}{\langle}
\newcommand{\ra}{\rangle}
\newcommand{\La}{L^{(\alpha)}(\Omega)}
%\newcommand{\Ref}[1]{\ref{#1}}
\newcommand{\apprle}{\lesssim}


%%%%%%%%% MULTIGRID FOR H(DIV) AND H(CURL)%%%
\newcommand{\diam}{\mbox{\rm diam\,}}
%\newcommand{\curl}{{\rm\bf  curl}}
%\newcommand{\grad}{{\rm\bf grad}}
%\DeclareMathOperator*{\supp}{supp}
\renewcommand{\div}{{\operatorname{div}}}
\newcommand{\sdiv}{\mbox{{\footnotesize div}}}
\newcommand{\scurl}{\mbox{{\footnotesize curl}}}%\newcommand{\eps}{\varepsilon}

\newcommand{\N}{{\mathbb N}}
%\renewcommand{\P}{{\mathbb{P}}}
%\newcommand{\R}{{\mathbb R}}
%\newcommand{\V}{{\mathbb V}}
\newcommand{\W}{{\mathbb W}}

\newcommand{\cA}{{\mathcal A}}
\newcommand{\cB}{{\mathcal B}}
\newcommand{\cC}{{\mathcal C}}
\newcommand{\cD}{{\mathcal D}}
\newcommand{\cE}{{\mathcal E}}
\newcommand{\cF}{{\mathcal F}}
\newcommand{\cH}{{\mathcal H}}
\newcommand{\cI}{{\mathcal I}}
\newcommand{\cJ}{{\mathcal J}}
\newcommand{\cL}{{\mathcal L}}
%\newcommand{\cM}{{\mathcal M}}
\newcommand{\cN}{{\mathcal N}}
\newcommand{\cO}{{\mathcal O}}
%\newcommand{\cP}{{\mathcal P}}
\newcommand{\cQ}{{\mathcal Q}}
\newcommand{\cS}{{\mathcal S}}
%\newcommand{\cT}{{\mathcal T}}
\newcommand{\cV}{{\mathcal V}}
\newcommand{\cW}{{\mathcal W}}

\newcommand{\bA}{{\boldsymbol A}}
\newcommand{\bB}{{\boldsymbol B}}
\newcommand{\bH}{{\boldsymbol H}}
\newcommand{\bL}{{\boldsymbol L}}
\newcommand{\bP}{{\boldsymbol P}}
\newcommand{\bQ}{{\boldsymbol Q}}
\newcommand{\bR}{{\boldsymbol R}}
\newcommand{\bS}{{\boldsymbol S}}
\newcommand{\bu}{{\boldsymbol u}}
\newcommand{\bv}{{\boldsymbol v}}
\newcommand{\bw}{{\boldsymbol w}}
\newcommand{\bp}{{\boldsymbol p}}
\newcommand{\bq}{{\boldsymbol q}}
\newcommand{\br}{{\boldsymbol r}}
%\newcommand{\bm}{{\boldsymbol m}}
\newcommand{\bsf}{{\boldsymbol f}}
\newcommand{\bn}{{\boldsymbol n}}
\newcommand{\bt}{{\boldsymbol t}}
\newcommand{\bx}{{\boldsymbol x}}
\newcommand{\bphi}{{\boldsymbol \phi}}
\newcommand{\bpsi}{{\boldsymbol \psi}}
%
\newcommand{\leqs}{\leqslant}
\newcommand{\nleqs}{\nleqslant}
\newcommand{\geqs}{\geqslant}
\newcommand{\ngeqs}{\ngeqslant}
%%%%%%%%%%%% end of multigrid for H(div) and H(curl)%%%%
